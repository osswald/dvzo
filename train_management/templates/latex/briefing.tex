\documentclass[titlepage]{article}

  % language setup
  \usepackage[utf8]{inputenc}
  \usepackage[ngerman]{babel}
  % use fancyhdr to style the header and footer
  \usepackage{geometry}
   \geometry{
   a4paper,
   left=15mm,
   right=15mm,
   top=25mm,
   bottom=35mm
   }
  \usepackage{fancyhdr}
  \usepackage{colortbl}
  \usepackage{color}
  \usepackage{multirow}
  \definecolor{red}{rgb}{0.929, 0.090, 0.294}


  \pagestyle{fancy}
  \usepackage{datetime}
  % make it easy to add figures and tables
  \usepackage{graphicx}
  
  \usepackage{tabularx}
  \usepackage{float}
  % setup the appendix
  \usepackage[toc,page]{appendix}
  % allow an emergency stretch of 1em to avoid oversize errors
  \emergencystretch=1em
  % removing the [noframe] will show a frame for each box
  \usepackage[noframe]{showframe}
  \usepackage{pageslts}


  \newdateformat{monthdayyeardate}{
    \THEDAY. \monthname[\THEMONTH] \THEYEAR}

  % setup fancy header
  \pagestyle{fancy}
  \fancyhf{}

  \fancyhead[ER,OL]{
  \includegraphics[height=3\baselineskip, scale=.2]{dvzo-logo.png}
  \begin{tabular}[b]{l}
    Dampfbahn-Verein Zürcher Oberland\\
    Postfach, 8494 Bauma, info@dvzo.ch, www.dvzo.ch\\
    CHE-104.143.432 MWST
  \end{tabular}
  }
  \addtolength{\headheight}{2\baselineskip}
  \addtolength{\headheight}{0.61pt}
  \rfoot{Seite \thepage von \lastpageref{LastPages}}
  \lfoot{\monthdayyeardate\today}

  \renewcommand{\familydefault}{\sfdefault}


  \begin{document}
  \pagenumbering{arabic}

  \section{Briefing \vspace{1em}
  {{- dayplanning.label -}} \vspace{1em} - \vspace{1em}
  {{- dayplanning.date.strftime('%d.%m.%Y') -}}}
  \begin{flushleft}
    \textbf{Allgemeines:}

    Der Rottenwagen ist aufgrund der Corona-Situation geschlossen. Christian Schlatter bringt um 08.00 Uhr Gipfeli.

    Für 11.30 und 12.30 Uhr haben Philipp und Christoph Pizzas organisiert. Bitte haltet die Abstandsregeln ein!

  \end{flushleft}

  \begin{flushleft}
  \textbf{Corona Briefing:}

    Das Corona Briefing mit Andrea findet um 08.15 und 09.15 Uhr beim Bahndiensthäuschen statt und
     ist für das Betriebspersonal obligatorisch!
  \end{flushleft}

  \begin{flushleft}
  \textbf{Tourenplan:}

    Der Tourenplan folgt separat.
  \end{flushleft}
  
  \begin{table}[htbp]
  \centering
  \begin{tabularx}{\textwidth}{|  X | }
    \hline
    
    \cellcolor{red}\textbf{ {{ train }} }  &  
    \\ \hline

    
        
        {{ vehicle.label }}

        
         & 


    
    \\ \hline
      \end{tabularx}
  \end{table}
  

  \begin{flushleft}
  \textbf{Personal}
  \end{flushleft}
  \begin{table}[htbp]
  
    \begin{tabular}{|l|l|l|l|}
      \hline Einsatzort & Funktion & Name & Telefon \\
      \hline
      
        
		
        
          
        
        \multirow{ {{ functions_table.row_count }} }{*}{ {{ train}} }
        
          & \multirow{ {{ persons|length }} }{*}{ {{ function }} }
          
            
              &
            
              & {{ person}} &
	    
              {{ person.mobile_phone.as_national }}
            
            \\
          
          \cline{2-4}
        
        \hline
      

      
        
        
	  
        
        \multirow{1}{*}{ {{ function_type }} }
        
          & \multirow{ {{ persons|length }} }{*}{ {{ function }} }
          
            
              &
            
              & {{ person}} &
	    
              {{ person.mobile_phone.as_national }}
            
            \\
          
          \cline{2-4}
        
        \hline
      
    \end{tabular}
  \end{table}

\end{document}
